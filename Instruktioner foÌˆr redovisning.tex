%%%%%%%%%%%%%%%%%%%%%%%%%%%%%%%%%%%%%%%%%%%%%%%%%%%%%%%%%%%%%%%%%%%%%%%%%%%%%%%%
%
%	Instruktioner för redovisning för E-sektionens teletekniska avdelning
%	Skriven av Viktor Ahlqvist
%	http://www.texempelvis.se
%
%	Tänkt att användas tillsammans med LuaTeX (lualatex).
%
%%%%%%%%%%%%%%%%%%%%%%%%%%%%%%%%%%%%%%%%%%%%%%%%%%%%%%%%%%%%%%%%%%%%%%%%%%%%%%%%

% Nag är ett paket som bör stå överst i alla dokument. Det tillför inga nya
% funktioner, men skriver varningar om man använder gamla, utdaterade,
% kommandon.
\RequirePackage[l2tabu, orthodox]{nag}

% Som dokumentklass används scrbook från KOMA-Script.
\documentclass[%
	a4paper,
	twoside,
	titlepage,
	swedish,
	DIV=7,
	final,
	]{scrbook}

%%% Generella paket %%%%%%%%%%%%%%%%%%%%%%%%%%%%%%%%%%%%%%%%%%%%%%%%%%%%%%%%%%%%

% Lägger till en uppsjö kommandon kopplade till importerad grafik, skalning, mm.
\usepackage{graphicx}

% Paket som rör tabeller
\usepackage{longtable}	% Lägger till 'longtable'-tabeller
\usepackage{xtab}		% Lägger till 'xtabular'-tabeller
\usepackage{booktabs}	% Utökar funktionaliteten hos linjerna i tabeller

% Scrpage2 från KOMA-Script för sidhuvuden och -fötter.
\usepackage{scrpage2}

% Todo-notes, att göra listor
\usepackage[obeyFinal]{todonotes}

% Anteckningar i marginalen
\usepackage{marginnote}
%%%%%%%%%%%%%%%%%%%%%%%%%%%%%%%%%%%%%%%%%%%%%%%%%%%%%%%%%%%%%%%%%%%%%%%%%%%%%%%%
	
%%% Språk %%%%%%%%%%%%%%%%%%%%%%%%%%%%%%%%%%%%%%%%%%%%%%%%%%%%%%%%%%%%%%%%%%%%%%

% Översätter fasta rubriker och ser till att rätt avstavning sker.
\usepackage{polyglossia}
\setdefaultlanguage{swedish}

% I engelska används punkt som decimalavskiljare och kommatecken för att 
% separera element i listor. Icomma löser detta på magiskt sätt.
% \usepackage{icomma}

% Byt multiplikationssymbol till punkt.
\def\ctimes\times					% "Sparar" krysset i \ctimes
\AtBeginDocument{\let\times\cdot}
%%%%%%%%%%%%%%%%%%%%%%%%%%%%%%%%%%%%%%%%%%%%%%%%%%%%%%%%%%%%%%%%%%%%%%%%%%%%%%%%

%%% Typsnitt, typografi samt utseende %%%%%%%%%%%%%%%%%%%%%%%%%%%%%%%%%%%%%%%%%%

% Ställ in typsnittet till Linux Libertine
\usepackage[oldstyle]{libertine}

% Ändra typsnitten i dokumentet
\setkomafont{disposition}{\normalfont}
\setkomafont{pagehead}{\normalfont}
\setkomafont{pagefoot}{\normalfont}

% Byt typsnitt som används i länkarna. Behöver defineras om efter hyperref har
% laddats.
\AtBeginDocument{\renewcommand\UrlFont{\normalfont}}

% Unicode-math är ett paket för att ge fullt unicode-stöd i matteläget och lätt
% att byta typsnitt.
% \usepackage[math-style=ISO,bold-style=ISO,vargreek-shape=unicode]{unicode-math}
% \setmathfont{Asana Math}

% Microtype, "Subliminal refinements towards typographical perfection"
\usepackage{microtype}

% Changepage, för att leka med marginaler och igenkänning om recto/verso.
\usepackage[strict]{changepage}
%%%%%%%%%%%%%%%%%%%%%%%%%%%%%%%%%%%%%%%%%%%%%%%%%%%%%%%%%%%%%%%%%%%%%%%%%%%%%%%%

%%% Egna kommandon och omdefineringar %%%%%%%%%%%%%%%%%%%%%%%%%%%%%%%%%%%%%%%%%%

% Fix för marginalerna
\usepackage{mparhack}
\usepackage[side,flushmargin]{footmisc}
\newcommand{\raggedmarginpar}[1]{%
	\marginpar[\raggedleft#1]{\raggedright#1}
}

% Stäng av numrering av rubriker
\setcounter{secnumdepth}{-1}
\setcounter{tocdepth}{1}

% Ta bort punkter i innehållsförteckningen
\makeatletter
	\renewcommand{\@dotsep}{10000} 
\makeatother

% Ändra figurtexten för tabeller
\renewcommand*{\tableformat}{%
  \tablename%~\thetable%
%  \autodot%
}

% Kommando för förkortningar
\newcommand{\acr}[1]{\textsc{#1}}

% Miljö för redovisningsinstruktioner i exemplen
\newenvironment{redovisning}%
{\vspace{\baselineskip}
% \marginnote[\baselineskip]{\textsc{löpande\\bokföring}}}%
\raggedmarginpar{\textsc{löpande\\bokföring}}}%
{\vspace{\baselineskip}}

% Årsbokslut
\newenvironment{bokslut}%
{\vspace{\baselineskip}
\raggedmarginpar{\textsc{årsbokslut}}}%
% {\marginnote[\baselineskip]{\textsc{årsbokslut}}}%
{\vspace{\baselineskip}}

% Kontoreferenser
\newcommand{\kref}[1]{\textbf{#1}}

% ;ängder som används för att få full sidbredd i \begin{addmargin}
\makeatletter
	\newlength{\overhang}
	\setlength{\overhang}{\marginparwidth}
	\addtolength{\overhang}{\marginparsep}
\makeatother

% Ändra sidhuvud och sidfot till något vi vill ha
\clearscrheadfoot
\ofoot[\pagemark]{\pagemark}
\pagestyle{scrheadings}
%%%%%%%%%%%%%%%%%%%%%%%%%%%%%%%%%%%%%%%%%%%%%%%%%%%%%%%%%%%%%%%%%%%%%%%%%%%%%%%%

%%% Dokumentdata %%%%%%%%%%%%%%%%%%%%%%%%%%%%%%%%%%%%%%%%%%%%%%%%%%%%%%%%%%%%%%%
\title{Instruktioner för redovisning}
\publishers{E-sektionens teletekniska avdelning}
\author{E-sektionens teletekniska avdelning}
\author{Viktor Ahlqvist}
%%%%%%%%%%%%%%%%%%%%%%%%%%%%%%%%%%%%%%%%%%%%%%%%%%%%%%%%%%%%%%%%%%%%%%%%%%%%%%%%

%%% hyperref behöver laddas sist %%%%%%%%%%%%%%%%%%%%%%%%%%%%%%%%%%%%%%%%%%%%%%%
\usepackage{hyperref}
\hypersetup{
  unicode,					% För att få unicode-kodade pdfsträngar.
  breaklinks,				% Tillåter att länkar radbryts.
  pdfauthor = {Viktor Ahlqvist och E-sektionens teletekniska avdelning},
  pdftitle = {Instruktioner för redovisning},
  pdfsubject = {Redovisning},% Ämne
  pdfborder = {0 0 0},
  bookmarksdepth = section,
  citecolor = DarkGreen,
  linkcolor = DarkBlue,
  urlcolor = DarkGreen,
}
%%%%%%%%%%%%%%%%%%%%%%%%%%%%%%%%%%%%%%%%%%%%%%%%%%%%%%%%%%%%%%%%%%%%%%%%%%%%%%%%


%%% Dokumentet %%%%%%%%%%%%%%%%%%%%%%%%%%%%%%%%%%%%%%%%%%%%%%%%%%%%%%%%%%%%%%%%%
% Här börjar det faktiska innehållet till dokumentet.
\begin{document}

	%%% Försättsblad och 'frontmatter' %%%%%%%%%%%%%%%%%%%%%%%%%%%%%%%%%%%%%%%%%
	\frontmatter

	\makeatletter
	\begin{titlepage}
		\begin{flushright}
			\vspace*{0.45\textheight}
			{\Large \@title \\}
			% {\large \@subtitle \\[1.5\baselineskip]}
			{\scshape \@author}
		\end{flushright}

	% \raggedleft
	% \vspace*{\baselineskip}
	% {\scshape \@author \\[0.4\textheight]}
	% {\Huge \@title \\[1.5\baselineskip]}
	% {\large \@subtitle}

	\end{titlepage}
	\makeatother

	\newpage
	\begin{addmargin*}[0em]{-\overhang}
		~\vfill
		\thispagestyle{empty}

		\textsc{Typsatt med luatex/latex och typsnittet linux libertine}\\[2\baselineskip]

		Detta verk är licensierat under en \emph{Creative Commons Erkännande 4.0 Internationell Licens.}

		Licensen innebär att du får dela och distribuera materialet i sin helhet eller i delar. Du får även göra anpassningar och ändringar i materialet så länge du hänvisar till källan. Licensen finns att läsa på \url{http://creativecommons.org/licenses/by/4.0/}\\[1.2\baselineskip]

		Den här skriften och koden som producerar den finns att tillgå på \url{https://github.com/vikahl/Instruktioner-redovisning/}. Hittar ni några felaktigheter eller förslag på förbättringar får ni gärna rapportera dessa där.
	\end{addmargin*}

	%%% FÖRORD
	% \chapter{Förord}
Lorem ipsum dolor sit amet, consectetur adipiscing elit. Proin et neque elit. Phasellus ornare nulla et neque congue ullamcorper. Praesent sed semper neque. Nullam pharetra feugiat nulla suscipit condimentum. In sodales turpis quis mauris varius rhoncus. Nullam euismod 
faucibus nibh sed molestie. Vestibulum ante ipsum primis in faucibus orci luctus et ultrices.

Posuere cubilia Curae; Sed eu nisl commodo, tincidunt urna quis, sagittis odio. Etiam ultricies sed magna nec adipiscing. Mauris ac orci ante. Nulla convallis rutrum est quis porttitor. Pellentesque habitant morbi tristique senectus et netus et malesuada fames ac turpis egestas. Proin quis hendrerit nisl, eget mollis erat. Nunc vitae pellentesque magna, at euismod leo.

Nulla facilisi. Mauris rhoncus ipsum justo, et tristique justo molestie non. Curabitur nibh lorem, scelerisque scelerisque lorem nec, pharetra bibendum dolor. Sed luctus odio sit amet tellus ultrices egestas. Donec convallis magna egestas, sagittis lorem a, tincidunt dui. 

Nullam lectus nisi, accumsan ac eleifend vel, consequat quis quam. Pellentesque blandit quam id elementum sollicitudin. Aliquam aliquet vel purus non aliquam. In vestibulum sem vel lacus lobortis, fermentum varius diam condimentum. Cras sit amet mattis neque. Mauris sit amet lacinia libero. Curabitur lacus tortor, vestibulum at lorem porttitor, tincidunt interdum tellus. Donec posuere molestie augue, a placerat tellus hendrerit at. Quisque orci augue, pulvinar quis euismod id, dictum eget elit. Donec congue magna quis vulputate sagittis.

\emph{-- Viktor Ahlqvist}
	
	\tableofcontents	% Skriver ut innehållsförteckning
	
	%%% Huvudinnehållet, 'main matter' %%%%%%%%%%%%%%%%%%%%%%%%%%%%%%%%%%%%%%%%%
	\mainmatter
	\part{Inledning}
	\chapter{En introduktion till bokföring}
Bokföring, att föra sina böcker, handlar om att föra en lista över alla affärshändelser som har hänt. Oavsett om det handlar om att man har köpt några pennor, tagit lån, investerat i maskiner eller betalat ut löner så ska detta föras upp i bolagets böcker.

Bokföringen finns till för att man ska kunna ha kontroll över sin ekonomi och kunna följa upp hur utvecklingen går. För en förening som \acr{eta} ligger också den löpande bokföringen till grund för de ekonomiska rapporterna som ska lämnas till föreningsstämman och kårens revisorer. Eftersom \acr{eta} är en förening med liten omsättning finns det inga lagkrav på att föra bok,\footnote{Föreningar som uppfyller något av nedanstående är skyldiga att föra bok.
	\begin{itemize}\itemsep1pt
		\item Tillgångarnas marknadsvärde överstiger 1,5 miljoner
		\item Föreningen bedriver närings\-verksamhet
		\item Föreningen är moderföretag i en koncern
	\end{itemize}
}
men utgifter ska redovisas och det rekommenderas att följa gängse bokföringsrutiner när redovisningen upprättas.

Eftersom det inte finns något lagstadgat bokföringskrav kan man undgå från en del av reglerna för att bokföringen ska underlättas. Man kan slå ihop transaktionerna och till exempel bara redovisa saldot i kassorna månadsvis. Huvudtanken bör dock vara att följa god redovisningssed i den mån det går.
% Fotnoter från http://wiki.sverok.se/wiki/Bokf%C3%B6ring,_introduktion

\section{Dubbel bokföring}
Den enklaste formen av bokföring är enkel bokföring, där varje transaktion skrivs upp på ett ställe. Köper man till exempel lödtenn för 100~kr skriver man upp lödtenn i kategorin \emph{Förbrukningsmateriel labb}. En personlig kassabok fungerar oftast på det här sättet.
Vid dubbel bokföring skrivs varje transaktion upp två gånger, man beskriver både var pengarna kommer ifrån och var de går. Med samma inköp på lödtenn för 100~kr hade man i dubbel bokföring noterat att pengarna kom från bankkontot och gick till förbrukningsmateriel.

Dubbel bokföring gör det mycket enklare att upptäcka fel och att göra avstämningar mellan olika konton. I stort sett samtliga affärssystem som finns för bokföring använder dubbel bokföring och i den resterande delen av den här texten kommer bara dubbel bokföring att användas.

\section{Verifikationer}
För varje affärstransaktion ska det finnas ett underlag som beskriver händelsen. Oftast är detta ett kvitto vid inköp, ett kontoutdrag eller en faktura. Den här handlingen kallas verifikation och för varje transaktion ska det finnas ett verifikat i bokföringen i den ordningen som transaktionerna ägt rum.

Finns det inga externa underlag som kvitton, till exempel vid bokföringsmässiga transaktioner kan man skriva en bokföringsorder.\todo{Här ska det in text}

\section{Konton}
När man pratar om konton i bokföring menar man inte alltid bankkonton. Ett konto i bokföringen är en redovisningspost, en kategori, snarare än ett konto på banken. Visserligen har föreningens bankkonto även ett konto i bokföringen men även till exempel förbrukningsinventarier, medlemsavgifter och inventarier har konton i bokföringen.

Man delar upp kontona i fyra olika kategorier, olika kontoslag:
\begin{itemize}\itemsep2pt
	\item tillgångar
	\item skulder
	\item intäkter
	\item kostnader
\end{itemize}

\clearpage

\subsection{Tillgångskonton}
\marginnote[]{
Konto: 1\,000--1\,999\\
\small
\begin{tabular}{ll}
	\textbf{debet}	& \emph{pengar in}\\
					& tillgångsökning\\
	\textbf{kredit} & \emph{pengar ut}\\
					& tillgångsminskning 
\end{tabular}
}
Tillgångskontona används för att bokföra sådant som föreningen äger. Detta innefattar dels likvida medel som pengar på bankkontot, kylkassan och liknande. I den mån föreningen äger inventarier eller maskiner som har ett bokfört värde så bokförs även det på bokföringskonto.

Är medlemmar eller andra personer skyldiga pengar har föreningen en fordran och detta skrivs upp som en tillgång. Om föreningen istället är skyldig en person pengar är detta en skuld, inte en negativ fordran, och ska bokföras på skuldkonton.

\subsection{Skuldkonton}
\marginnote[]{
Konto: 2\,000--2\,999\\
\small
\begin{tabular}{ll}
	\textbf{debet}	& \emph{skuld minskar}\\
	\textbf{kredit}	& \emph{skuld ökar}
\end{tabular}
}
Skuldkontona används för de skulder föreningen har. Detta kan dels innehålla skulder till banker och leverantörer så väl som skulder till medlemmar. I företag går ägarna in med pengar när man startar företaget och företaget har då en skuld till ägarna. \acr{eta} har tidigare inte haft några skulder.

\subsection{Intäktskonton}
\marginnote[]{
Konto: 3\,000--3\,999 (8\,000)\\
\small
\begin{tabular}{ll}
	\textbf{debet}	& \emph{felkorrigering}	\\
					& inkomstminskning\\
	\textbf{kredit}	& \emph{får pengar}\\
					& inkomstökning
\end{tabular}
}
Medlemsavgifter, försäljning av mat och auktionsgods är exempel på intäkter till föreningen som bokförs på 3\,000-konton. Dessa konton används bara när föreningen har en faktiskt intäkt, om pengar flyttas från kassan till bankkontot är det ingen intäkt utan bara en flytt mellan två tillgångskonton.

Man pratar sällan om negativa intäkter och debet-delen används nästan enbart för felkorrigeringar i det här kontoslaget.

\subsection{Kostnadskonton}
\marginnote[]{
Konto: 4\,000--8\,999\\
\small
\begin{tabular}{ll}
	\textbf{debet}	& \emph{varor köps}\\
					& utgiftsökning\\
	\textbf{kredit} & \emph{felkorrigering}\\
					& utgiftsminskning
\end{tabular}
}
Kostnader är allt ifrån inköp av gem och pennor så väl som gas till svetsen eller datorutrustning. Alla inköp som inte har ett bestående värde och är att betrakta som en investering är kostnader för föreningen. \emph{Bestående värde} är inte helt entydigt definierat, men om varan kan säljas vidare kan man anta att den har ett bestående värde. Värt att notera är att normalt bokförs till exempel datorutrustning som förbrukningsmateriel och har inte något bestående värde på samma sätt som dyrare maskiner.

\acr{eta} har tidigare inte gjort några avskrivningar på maskiner och har inte heller krav på sig att göra så. \todo{tror jag i alla fall} Därför kan kostnadskonton används för nästan alla inköp.

\subsection{Kontoplan}
En kontoplan är en lista med konton som har dels ett nummer och ett namn. Det finns flera standardiserade kontoplaner, till exempel \acr{bas} eller \acr{eu}. 
Även om man kan hitta på en helt egen kontoplan så är det vanligaste att man följer en standardiserad plan och sedan lägger till eller tar bort konton som man behöver. \acr{eta} har till exempel inga anställda och äger inte heller fastigheter, så konton som rör det behövs inte.

Kontoplanen som är föreslagen i den här skriften är baserad på \acr{bas}-kontoplanen och sedan anpassad för \acr{eta}s behov. Detta gör det lätt att sätta sig och anpassa kontoplanen om behov uppstår.

\section{Kostnadsställe}
Ett kostnadsställe är en avgränsad enhet eller verksamhet inom organisationen och på \acr{eta} kan det till exempel vara Värkstaden, labbet eller auktionen. Det finns inget krav på att använda kostnadsställen och det går utmärkt att bokföra utan dessa. De kan dock underlätta att gruppera utgifter i organisatoriska grupper också och inte bara efter kontotyp, speciellt när man ska göra budgetuppföljning. 

När man använder kostnadsställen anger man dels kostnadskonto och dels kostnadsställe.

Kostnadsställen kan ibland ha ett annat namn i redovisningsprogrammet, i Visma administration heter det till exempel resultatenhet.

\section{Kontant-- eller fakturametoden}
I större företag används oftast fakturametoden för att bokföra fakturor. Metoden innebär att fakturor bokförs två gånger, dels när fakturan tas emot och dels när den betalas. När fakturan tas emot bokförs den som en skuld eller fordran och när fakturan faktiskt är betald bokför man bort skulden mot bankkontot.

Mindre företag och föreningar \footnote{Nettoomsättning under 3 miljoner kronor och utan krav på att upprätta årsbokslut} kan istället använda kontantmetoden. Då bokförs fakturorna först då de betalas och man noterar alltså inte att föreningen har en skuld eller fordran. Detta underlättar bokföringen, men gör att man inte har samma insyn i nuläget. För \acr{eta} är inte värdet på fakturorna så stort att behov av större insyn finns. 


\section{Bokföringsexempel}
\todo{Uppdatera det här avsnittet med rätt kontonummer och kostnadsställen}
Nedan följer några exempel på verifikat och bokförda transaktioner.

\subsection{Inköp av förbrukningsmateriel}
I det här exemplet har det köps in lödtenn och annan förbrukningsmateriel för att använda i labbet.
Betalningen har skett med kontokort och totala summan av de inköpta varorna är 285~kr. Då sakerna ska användas i labbet anger vi också labbets kostnadsställe.

I och med den här betalningen har saldot på bankkontot minskats, och bankkontot ska därför krediteras. För att summan ska bli lika stora i båda kolumnerna måste kostnadskontot således debeteras.

\begin{longtable}{llrr}
	\caption{Exempel: Bokföra materielinköp}\\
	Konto	& Kontonamn					& Debet		& Kredit\\ \toprule
	1930	& Företagskonto				& 			& 285~kr\\
	5410.10	& Förbrukningsinventarier	& 285~kr	& \\ \bottomrule
			& Summa						& 285~kr	& 285~kr
\end{longtable}

\subsection{Inbetalning av medlemsavgift}
Nu har \acr{eta} istället fått pengar i form av en medlem som har betalat in sin medlemsavgift. Då \acr{eta} bara har en typ av medlemsavgift finns det ingen anledning att särskilja på dessa med till exempel kostnadsställen.

I det här exemplet sätts pengar in på bankkontot, och det ska alltså debeteras. För att summan av debet och kredit ska bli lika stora måste således kontot för medlemsavgifter krediteras.

\begin{longtable}{llrr}
	\caption{Exempel: Bokföra medlemsavgift}\\
	Konto	& Kontonamn					& Debet		& Kredit\\ \toprule
	1930	& Företagskonto				& 100~kr	& \\
	3210	& Medlemsavgifter			& 			& 100~kr\\ \bottomrule
			& Summa						& 100~kr	& 100~kr
\end{longtable}

\subsection{Inköp av hylla till förrådet}
% I det här fallet behövs en ny hylla till förrådet. När man köper in inventarier redovisas de normalt som tillgångar och skrivs av under sin livslängd. För inventarier av mindre värde\footnote{Anskaffningsvärde som understiger ett halvt prisbasbelopp, 22\,000~kr (2014)} kan utgiften istället kostnadsföras som förbrukningsinventarier. \acr{Eta} får dock alltid bokföra inventarieinköp som direkta kostnader, men det kan vara bra att känna till de allmänna principerna också.

Hyllan i exemplet kostade 5\,200~kr och har betalats med kontokort. Saldot på bankkontot minskade, alltså ska det kontot debeteras.

\begin{longtable}{llrr}
	\caption{Exempel: Bokföra inventarieinköp}\\
	Konto	& Kontonamn					& Debet		& Kredit\\ \toprule
	1930	& Företagskonto				& 			& 5\,200~kr\\
	5411.20	& Förbrukningsinventarier	& 5\,200~kr	& \\ \bottomrule
			& Summa						& 5\,200~kr	& 5\,200~kr
\end{longtable}

\subsection{Inköp till kylen}
% När varor till kylen köps in får \acr{eta} ett varulager som har ett värde. Man kan tänka sig att man bokför varuinköpet som en kostnad och sedan varuförsäljningen som en inkomst, men det är då svårt att se om man

När det gäller inköp av varor kan de bokföras på olika sätt. I ett företag hade man bokfört inköpet som att värdet på varulagret ökade, men för \acr{eta} tror jag det blir för svårt att hålla i en rutin med varulager när det gäller kylarna. De fungerar ju på sådant sätt att medlemmarna själva betalar och det är svårt att veta exakt vad som säljs (även om man kan räkna på lagervärde). Ibland ges delar av lagret bort (mutgodis) och då måste det bokföras, vilket antagligen blir onödigt komplicerat.

Därför bokförs inköpen till kylen som en kostnad, och när försäljningen sedan sker uppstår en inkomst.
\begin{longtable}{llrr}
	\caption{Exempel: Bokföra inventarieinköp}\\
	Konto	& Kontonamn					& Debet		& Kredit\\ \toprule
	1930	& Företagskonto				& 			& 7\,869~kr\\
	4011	& Inköp av varor			& 7\,869~kr	& \\ \bottomrule
			& Summa						& 7\,869~kr	& 7\,869~kr
\end{longtable}


\subsection{Försäljning via kylen}



\subsection{Insättning av kylkassa}

	\part{Redovisningsråd}
	\chapter{Redovisning}


				\section{Auktionshämtning}
				\emph{se transporter}
			
				\section{Auktionssortering, mat}
				\emph{se medlemsförmåner}
			
				\section{Bestick}
				\emph{se förbrukningsmateriel}
			
				\section{Diskmedel}
				\emph{se förbrukningsmateriel}
			
				\section{Farnell}
				\emph{se förmedling av varor}
			
				\section{Förbrukningsinventarier}
				\emph{se inventarier}
			
				\section{Förbrukningsmateriel}
				
				I förbrukningsmateriel ryms all inköpt materiel som inte har ett bestående värde och som förr eller senare förbrukas.
		
		Här ingår alltså till exempel pennor och papper, men även lödtenn och komponenter. När det gäller verktyg och gerät får en bedömning från fall till fall göras. Man kan anse att borr som köps in är förbrukningsmateriel eftersom det inte har något bestående värde och slits ut relativt snabbt. När det gäller borrmaskiner däremot, kan man anse att de är tänkta att användas under en längre tid och därmed bör räknas som inventarier. Om osäkerhet uppstår är det viktigare att kostnaden hamnar på rätt kostnadsställe snarare än om det är förbrukningsmateriel eller en inventarie.
		
		Förbrukningsmateriel som har en mer permanent karaktär och är tydligt kopplat till lokalen kan istället bokföras på kontot för lokalkostnader.
		
					\begin{redovisning}
						Förbrukningsmateriel bokförs på konto \kref{5400} och tillhörande kostnadsställe.
					\end{redovisning}
				
				\section{Förmedling av varor}
				
				
				\section{Gerät}
				\emph{se inventarier}
			
				\section{Instrument}
				\emph{se inventarier}
			
				\section{Inventarier}
				
				
					\todo{Får ETA verkligen skriva av allt direkt? Hitta en tillförlitlig källa på det!}
				När man köper in inventarier redovisas de normalt som anläggningstillgångar och skrivs av under sin livslängd. För förbrukningsinventarier, inventarier av mindre värde\footnote{Anskaffningsvärde exklusive moms som understiger ett halvt prisbasbelopp, 22\,000~kr (2014)} och korttidsinventarier, kan utgiften istället dras av direkt, så kallat direktavdrag. Inventarier som har ett naturligt samband ska bedömas gemensamt, samma sak gäller för inventarier som är en del av en större inventarieinvestering. Man kan alltså inte 'dela upp' inventarierna för att få ner värdet.
		
					\begin{redovisning}
						
		För inventarier som understiger ovan nämnda värde bokförs kostnaden direkt på kontot för inventarier. För inventarier som har en tydlig koppling till lokalen, till exempel lysrör eller elcentral till förrådet, kan istället bokföringen ske på kontot för lokalutgifter.
		
		Glöm inte att ange rätt kostnadsställe för inventarierna.
		
					\end{redovisning}
				
				\section{Mat}
				\emph{se medlemsförmåner}
			
				\section{Medlemsavgifter}
				
				Medlemsavgifter utgör huvudinkomsten för \acr{eta} och bokförs på därtill förenligt intäktskonto.
		
		Det förekommer ibland att medlemmar betalar för flera år samtidigt och man kan då diskutera om man ska periodisera intäkten över de åren eller om hela intäkten ska bokföras direkt. Detta förekommer dock ganska sällan och påverkar inte resultatet nämnvärt. Det finns dessutom risk att informationen inte förs vidare till nästa års styrelse så eventuella vinningar är mindre än de eventuella extra komplikationerna.
		
		Eftersom medlemsavgifterna inte följer verksamhetsår kan man på samma sätt diskutera huruvida man ska periodisera alla avgifter över årsbrytet, men då antalet medlemmar är ganska konstant medför det här, precis som ovan nämnda punkt, mycket extraarbete för lite resultat.
		
					\begin{redovisning}
						Medlemsavgifter krediteras kontot \kref{3100}. Kostnadsställe behöver inte anges eftersom det bara finns en typ av inkomst.
					\end{redovisning}
				
				\section{Medlemsförmåner}
				
				
					\todo{Man skulle kunna kalla den här för \emph{föreningskvällar} som motsvarande kostnader har bokförts på tidigare. Det gör det dock inte tydligt att även kaffe, mutgodis och liknande bör hamna på det här kontot. Det är trots allt intressant att följa upp hur mycket pengar som läggs på förmåner till medlemmarna (till exempel kaffe).}
				Medlemmar i \acr{eta} har flera förmåner efter att man har gått med i föreningen. Man har tillgång till komponentförrådet, lokalen och har också tillgång till kostnadsfritt kaffe och tisdagsfika.
		
		Utgifter för komponentbaren och lokalen bokförs på sina respektive konton, under medlemsförmåner räknas de förmåner som inte strikt har med verksamheten. Här innefattas bland annat \emph{tisdagsfika, fritt kaffe, mat i samband med sorteringar} och \emph{mutgodis}. Exakt vad gränsen går mellan vanlig verksamhet och förmåner är inte entydigt definierat utan kassören får göra en bedömning i gränsfallen.
		
					\begin{redovisning}
						Medlemsförmåner bokförs på konto \kref{7600} med tillhörande kostnadsställe.
					\end{redovisning}
				
				\section{Mutgodis}
				\emph{se medlemsförmåner}
			
				\section{OBS-konto}
				
				I arbetet med bokföringen händer det ibland att det saknas underlag för en transaktion eller att det inte går att avgöra hur den ska bokföras. Gängse bokföringsregler \todo{källa behövs} säger att transaktioner ska bokföras i den ordning de har uppstått. För att inte bokföringsabetet ska avstanna helt när vid tidigare nämnda situationer kan ett \acr{obs}-konto, observationskonto, användas för att bokföra kostnaden på. Man debiterar då kostnaden på \acr{obs}-kontot och när underlaget sedan har kommit fram krediterar man \acr{obs}-kontot och debiterar rätt konto.
		
		Vid årsbokslutet ska saldot på \acr{obs}-kontot vara noll.
		
					\begin{redovisning}
						Okända kostnader debiteras kontot \kref{2999} och när sedan kostnaden reds ut krediteras \kref{2999} och rätt konto debiteras.
					\end{redovisning}
				
					\begin{bokslut}
						Vid verksamhetsårets slut måste saldot på \acr{obs}-kontot vara noll. Om det finns kostnader man inte kan reda ut får en diskussion med revisorerna och styret tas för att komma fram hur det ska bokföras. Saldot \kref{6900} kan användas för kostnader som inte passar in på något annat konto.
					\end{bokslut}
				
				\section{Tallrikar}
				\emph{se förbrukningsmateriel}
			
				\section{Tisdagsfika}
				\emph{se medlemsförmåner}
			
				\section{Transporter}
				
				Speciellt i samband med auktionen förekommer det många transporter och därmed också utgifter för dessa. Transporter av varor förekommer även under resten av året, då det till exempel handlas till kylen.
		
		Dessa transporter varierar mellan att vara extern inhyrda fordon, fordon hyrda av kåren och medlemmars egna fordon som lånas ut. I kontoplanen finns ett konto för alla typer av transporter, oavsett om det är kostnader för egna transporter eller inhyrt fraktbolag.
		
		I transportkostnaden inräknas alla utgifter som direkt rör transporten, till exempel drivmedel, parkeringskostnader eller hyra av fordon.
		
		Notera att post inte räknas som transport utan istället \kref{6250}.
		
					\begin{redovisning}
						Transporter bokförs på kostnadskonto \kref{5700} med ett kostnadsställe relevant för aktiviteten. Är det en transport som inte rör någon specifik kategori kan kostnadsställe utelämnas, men är det till exempel inköp av kylvaror ska dessa noteras på rätt kostnadsställe.
		
		Om transporten rör flera kostnadsställen kan antingen kostnaden delas upp eller bokföras på ett av dessa, beroende på hur stor kostnaden är.
					\end{redovisning}
				
	
	\part{Kontoplan}
	\chapter{Kontoplan}

Kontoplanen som redovisas här är ett förslag på kontoplan baserad på primärt \acr{bas}~2014 kontoplan 1 med stöd av kontoplan 2 där det behövts. \footnote{Kontoplanerna finns att tillgå gratis på \acr{bas}-gruppens hemsida, \url{www.bas.se}}

Alla konton i \acr{bas}-kontoplanen är inte användbara för \acr{eta} och det är generellt sällan man använder kontoplanen rent av. 
Därför har vissa konton lagts till och många har inte tagits med. En del konton har bytt namn, och i en del fall så står båda namnen med i listan nedan \todo{Varför?}. Med största sannolikhet kommer det komma situationer där ytterligare konton behövs, men i så fall är det bara att skapa dessa på rätt plats i listan. Ta i sådant fall hjälp av \acr{bas} kontoplanen om osäkerhet finns över var de passar in.


\section{Tillgångskonton}
\todo{Eftersom man inte kan ha marginalanteckningar i tabeller står 'todonotes' nedan i kursiv stil.}
\begin{minipage}{\fullwidthlength}
	\begin{longtable}[l]{l p{0.4\linewidth} p{0.5\linewidth}}
		Konto	&	Namn								& Kommentar \\ \toprule \endhead
		1510	&	Fordringar medlemmar \newline \emph{Kundfordringar} & Pengar som medlemmar är skyldiga \acr{eta}\\
		1515	&	Osäkra fordringar medlemmar \newline \emph{Osäkra kundfordringar}	& Pengar som medlemmar är skyldiga men där det är osäkert om man får igen. \newline Se ”Farnell” för exempel.\\
		1600	&	Övriga fordringar					& Andra personer än medlemmar som är skyldiga föreningen pengar.\\
		1910	&	Kassa								& För kassorna som förvaras i lokalen\\
		1911	&	Lokalkassa							& De olika kassakontona som ligger framme i lokalen.\\
		1930	&	Checkkonto \newline \emph{Företagskonto/checkkonto/affärskonto}	& Bankkonto\\
	\end{longtable}
\end{minipage}

\section{Skuldkonton}
\begin{minipage}{\fullwidthlength}
	\begin{longtable}[l]{l p{0.4\linewidth} p{0.5\linewidth}}
		Konto	&	Namn								& Kommentar \\ \toprule \endhead
		2440	&	Leverantörsskulder					& Skulder föreningen har till medlemmar eller leverantörer. \\
		2999	&	OBS-konto							& I den löpande bokföringen händer det ibland att det saknas underlag för en transaktion eller att det inte går att avgöra hur den ska bokföras. För att inte bokföringsarbetet ska avstanna använder men ett observationskonto som man sedan bokför från när underlaget är klart. Det här kontot ska nollställas vid årsbokslut. \emph{Se 'OBS-konto'}.
	\end{longtable}
\end{minipage}


\section{Intäktskonton}
\begin{minipage}{\fullwidthlength}
	\begin{longtable}[l]{l p{0.4\linewidth} p{0.5\linewidth}}
		Konto	&	Namn								& Kommentar \\ \toprule \endhead
		3100	&	Medlemsavgifter \newline \emph{Momsfria intäkter} \\
		3610	&	Försäljning av varor				& Försäljning av material och materiel i lokalen. Inkluderar kyl, laminat och liknande varor som köps in och sedan säljs efter hand.\\
		3611	&	Förmedling av varor					& Förmedling av varor, Farnell-beställningar, Atmel och dylikt\\
		3612	&	Försäljning auktion \emph{Behöver detta ett eget konto? Egentligen inte, men det kanske underlättar? Försäljningen här har ju väldigt lite att göra med resterande försäljning och för att särskilja auktionskafé och utrop kanske detta behövs?} & Auktionen\\
		3740	&	Öres-- och kronutjämning			& Specialkonto som används för öresutjämning\\
		3900	&	Övriga intäkter						& För intäkter som inte täcks av något annat konto
	\end{longtable}
\end{minipage}

\section{Kostnadskonton}
\begin{minipage}{\fullwidthlength}
	\begin{longtable}[l]{l p{0.4\linewidth} p{0.5\linewidth}}
		Konto	&	Namn								& Kommentar \\ \toprule \endhead
		4000	&	Inköp av varor						& Inköp av varor som säljs i lokalen, t.ex. kyl, laminat och dylikt.\\
		5000	&	Lokalkostnader						& Utgifter tillhörande lokalerna, fasta installationer.\\
		5400	&	Förbrukningsmateriel				& Förbrukningsmateriel, lödtenn, pennor och papper.\\
		5410	&	Inventarier \newline \emph{Förbrukningsinventarier} & Inköp av inventarier som inte ska skrivas av\emph{Skriv varför}\\
		5700	&	Frakter och transporter				& Alla kostnader för transporter, t.ex. auktionshämtning. Slår ihop kontogrupperna 56 och 57 då behov inte anses finnas för att särredovisa egen transport (t.ex. drivmedel) och externt inköpt transport (frakt) i två olika grupper.\\
		5900	&	Reklam och PR						& All form av medlemsrekryterande verksamhet.\\
		6000	&	Övriga försäljningskostnader		& Kostnader som uppstår i samband med försäljning av varor och förmedling av sådana.\\
		6210	&	Telekommunikation					& Telefonen som står i lokalen.\\
		6250	&	Postbefordran						& Post och porto\\
		6310	&	Försäkringar						& \acr{Eta}s försäkringar.\\
		6570	&	Bankkostnader						& Bankkonto kostar pengar.\\
		6900	&	Övriga kostnader					& Det kommer alltid finnas kostnader som inte täcks av de andra kontona.\\
		6970	&	Tidningar, tidskrifter och facklitteratur & Biblioteket, Kalle Anka och liknande.\\
		7600	&	Förmåner medlemmar					& Allt som ges till medlemmarna, kaffe, tisdagsfika och medlemsgrillningar.\\[1.5\baselineskip]
		\multicolumn{3}{l}{Finansiella och andra inkomster/intäkter och utgifter/kostnader}\\ \midrule
		8310	&	Ränteintäkter						& Räntan som fås på t.ex. bankkontot.\\
		8999	&	Årets resultat\emph{Ordna med motkonto. Förklara.}\\
	\end{longtable}
\end{minipage}


\chapter{Kostnadsställen}

Kostnadsställen används för att gruppera intäkter och kostnader till en avgränsad enhet eller verksamhet. Kostnadsställen är inget krav, men underlättar väldigt mycket för uppföljning.

Till skillnad mot kontoplanen som bör vara ungefär lika från år till år kan kostnadsställena uppdateras om verksamheten ändras eller nya områden tillkommer. Man kan till exempel vilja ha kostnadsställen för vissa större projekt eller jubileum för att kunna följa upp utgifterna bättre. Vissa redovisningsprogram har ytterligare funktioner för att gruppera utgifter, till exempel i projekt, men detta diskuteras inte i den här skriften.

Det underlättar budgetarbetet om kostnadsställena har en tydlig koppling till budgeten. Till exempel kan man budgetera på varje kostnadsställe eller gruppera ihop dessa.

Grundtanken är att \emph{varje kostnad ska noteras på ett kostnadsställe}. Inkomster bör också noteras i de fall det behövs, till exempel vid försäljning av varor.

\begin{longtable}[l]{l p{0.3\linewidth} p{0.6\linewidth}}
	Kostnadsställe	&	Namn								& Kommentar \\ \toprule \endhead
	\multicolumn{3}{l}{1--9 Föreningen}\\
	1				&	Administration						& Utgifter som krävs för föreningsarbetet. Bokföringsprogram, bankkonton och pennor. Bör finnas som en budgetpost och därmed också kostnadsställe.\\
	\multicolumn{3}{l}{10--19 Verksamheten}\\
	10				&	Lokalen\\
	11				&	Värkstaden\\
	12				&	Labbet\\
	13				&	Amatörradio\\
	19				&	Övrig verksamhet					& Övrigt direkt relaterat till verksamheten.\\
	\multicolumn{3}{l}{20--29 Auktionen}\\
	20				&	Auktion								& Allt som rör auktionen. Vill man budgetera på mer detaljnivå kan man lägga till konton för till exempel sorteringar, arbete innan och liknande.\\
	\multicolumn{3}{l}{30--39 Inkomster}\\
	30				&	Kylen								& Den löpande försäljningen av fika och kylvaror.\\
	31				&	Öppen försäljning lokalen			& ”Öppen försäljning”, alltså försäljning där medlemmarna själva tar varan och betalar. Intressant med ett eget kostnadsställe för att räkna på svinn.\\
	32				&	Övrig försäljning					& All annan försäljning som \acr{eta} gör.\\
	\multicolumn{3}{l}{40--49 Övriga kostnader}\\
	40				&	Övriga kostnader					& För budgetposten ”övriga kostnader”.\\
	\multicolumn{3}{l}{50--79 \emph{Ej använda}}\\
	\multicolumn{3}{l}{80--99 Rörande projekt}\\
	80				&	ETAprojekt							& Projekt som inte kan debiteras ett verksamhetsområde.\\
	81--99			&	\emph{Reserverat för specifika projekt} & Reserverat för specifika projekt.\\
\end{longtable}

	% Bilagor
	\backmatter
	\appendix
	\part{Bilagor}
	\chapter{Bilagor}

Det här kapitlet innehåller en samling bilagor, bland annat tabeller och lathundar som är tänkta att användas under redovisningsarbetet.


		
		% Genererad från kontoplan.xml med kontoplan-bilaga-latex.xsl.
		\begin{adjustwidth}{}{-8em}
			\section{Kontoplan}
			\label{sec:bilaga:kontoplan}
			\tablefirsthead{Konto	&	Namn \\\toprule}
			\begin{xtabular}[l]{l l}
				1510 & Fordringar medlemmar \\
				1515 & Osäkra fordringar medlemmar \\
				1600 & Övriga fordringar \\
				1910 & Kassa \\
				1911 & Lokalkassa \\
				1930 & Checkkonto \\
				2440 & Leverantörsskulder \\
				2999 & OBS-konto \\
				3100 & Medlemsavgifter \\
				3610 & Försäljning av varor \\
				3611 & Förmedling av varor \\
				3612 & Försäljning auktion \\
				3740 & Öres- och kronutjämning \\
				3900 & Övriga intäkter \\
				4000 & Inköp av varor \\
				5000 & Lokalkostnader \\
				5400 & Förbrukningsmateriel \\
				5410 & Inventarier \\
				5700 & Frakter och transporter \\
				5900 & Reklam och PR \\
				6000 & Övriga försäljningskostnader \\
				6210 & Telekommunikation \\
				6250 & Postbefordran \\
				6310 & Försäkringar \\
				6570 & Bankkostnader \\
				6900 & Övriga kostnader \\
				6970 & Tidningar, tidskrifter och facklitteratur \\
				7600 & Förmåner medlemmar \\
				8310 & Ränteintäkter \\
				8999 & Årets resultat \\
				
			\end{xtabular}
		\end{adjustwidth}

		
	
\end{document}