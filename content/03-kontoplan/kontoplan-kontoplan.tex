
		% Genererad från kontplan.xml med kontoplan-latex.xsl
		
		\begin{adjustwidth}{}{-8em}
			\section{Tillgångskonton}
			\label{sec:kontoplan:tillgång}
			\tablefirsthead{Konto	&	Namn	&	Kommentar	\\\toprule}
			\begin{xtabular}[l]{l p{0.4\linewidth} p{0.6\linewidth}}
				1510 & Fordringar medlemmar &  Pengar som medlemmar är skyldiga ETA \\
				1515 & Osäkra fordringar medlemmar & Pengar som medlemmar är skyldiga men där det är oklart om man kommer få tillbaka. \\
				1600 & Övriga fordringar & Andra personer än medlemmar som är skyldiga föreningen pengar. \\
				1910 & Kassa & För kassorna som förvaras inlåst lokalen \\
				1911 & Lokalkassa & För kassorna som medlemmarna har tillgång till. \\
				1930 & Checkkonto & Bankkontot \\
				
			\end{xtabular}
		\end{adjustwidth}
		
		\clearpage
		\begin{adjustwidth}{-8em}{}
			\section{Skuldkonton}
			\label{sec:kontoplan:skuld}
			\tablefirsthead{Konto	&	Namn	&	Kommentar	\\\toprule}
			\begin{xtabular}[l]{l p{0.4\linewidth} p{0.6\linewidth}}
				2440 & Leverantörsskulder & Skulder föreningen har till medlemmar eller leverantörer \\
				2999 & OBS-konto & I den löpande bokföringen händer det ibland att det saknas underlag för en transaktion eller att det inte går att avgöra hur den ska bokföras. För att inte bokföringsarbetet ska avstanna använder men ett observationskonto som man sedan bokför från när underlaget är klart. \\
				
			\end{xtabular}
		\end{adjustwidth}
	
		\begin{adjustwidth}{-8em}{}
			\section{Intäktskonton}
			\label{sec:kontoplan:intäkt}
			\tablefirsthead{Konto	&	Namn	&	Kommentar	\\\toprule}
			\begin{xtabular}[l]{l p{0.4\linewidth} p{0.6\linewidth}}
				3100 & Medlemsavgifter &  \\
				3610 & Försäljning av varor & Försäljning av materiel och varor i lokalen. Inkluderar kyl, laminat och liknande varor som köps in och säljs löpande. \\
				3611 & Förmedling av varor & Vinst som föreningen gör på förmedling av varor, där medlemmar köper varor via ETA (och inget lager finns). Till exempel Farnell, Atmel och liknande. \\
				3612 & Försäljning auktion & Försäljningen under auktionen \\
				3740 & Öres- och kronutjämning & Specialkonto som används för öresutjämning \\
				3900 & Övriga intäkter & Saker som helt enkelt inte täcks av något annat konto. \\
				8310 & Ränteintäkter & Räntan som fås på bankkontot. \\
				
			\end{xtabular}
		\end{adjustwidth}
		
		\clearpage
		\begin{adjustwidth}{}{-8em}
			\section{Kostnadskonto}
			\label{sec:kontoplan:kostnad}
			\tablefirsthead{Konto	&	Namn	&	Kommentar	\\\toprule}
			\begin{xtabular}[l]{l p{0.4\linewidth} p{0.6\linewidth}}
				4000 & Inköp av varor & Inköp av varor som säljs i lokalen: mat, laminat och dylikt. \\
				5000 & Lokalkostnader & Kostnader som rör fasta installationer i lokal. \\
				5400 & Förbrukningsmateriel & Lödtenn, pennor, papper, städutrustningen och all annan materiel som inte har bestående värde. \\
				5410 & Inventarier & Inköp av inventarier som inte ska skrivas av. \\
				5700 & Frakter och transporter & Alla kostnader för transporter, till exempel auktionshämtning. Slår ihop kontogrupperna 56 och 57, då det inte finns någon anledning att särredovisa egna transporter (t.ex. drivmedel) och externa transporter (frakt) i olika grupper. Uppstår behovet kan man ändra till två kontogrupper. \\
				5900 & Reklam och PR & All form av medlemsrekryterande verksamhet. \\
				6000 & Övriga försäljningskostnader & Kostnader som uppstår i samband med försäljning och förmedling av varor. \\
				6210 & Telekommunikation & Telefonen som står i lokalen. \\
				6250 & Postbefordran & Post och porto. Dock ej kuvert, det är förbrukningsmateriel. \\
				6310 & Försäkringar &  \\
				6570 & Bankkostnader & Avgifter till banken för till exempel konto och bankkort. Även transaktionskostnader. \\
				6900 & Övriga kostnader &  \\
				6970 & Tidningar, tidskrifter och facklitteratur & Biblioteket, prenumeration på Kalle Anka och liknande. \\
				7600 & Förmåner medlemmar & Allt som ges till medlemmar, så som kaffe, tisdagsfika och medlemsträffar. \\
				8999 & Årets resultat & Förklara och ordna med motkonto. \\
				
			\end{xtabular}
		\end{adjustwidth}

	