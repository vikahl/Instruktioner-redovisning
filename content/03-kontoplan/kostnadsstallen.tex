Kostnadsställen används för att gruppera intäkter och kostnader till en avgränsad enhet eller verksamhet. Kostnadsställen är inget krav, men underlättar väldigt mycket för uppföljning.

Till skillnad mot kontoplanen som bör vara ungefär lika från år till år kan kostnadsställena uppdateras om verksamheten ändras eller nya områden tillkommer. Man kan till exempel vilja ha kostnadsställen för vissa större projekt eller jubileum för att kunna följa upp utgifterna bättre. Vissa redovisningsprogram har ytterligare funktioner för att gruppera utgifter, till exempel i projekt, men detta diskuteras inte i den här skriften.

Det underlättar budgetarbetet om kostnadsställena har en tydlig koppling till budgeten. Till exempel kan man budgetera på varje kostnadsställe eller gruppera ihop dessa.

Grundtanken är att \emph{varje kostnad ska noteras på ett kostnadsställe}. Inkomster bör också noteras i de fall det behövs, till exempel vid försäljning av varor.

\begin{longtable}[l]{l p{0.3\linewidth} p{0.6\linewidth}}
	Kostnadsställe	&	Namn								& Kommentar \\ \toprule \endhead
	\multicolumn{3}{l}{1--9 Föreningen}\\
	1				&	Administration						& Utgifter som krävs för föreningsarbetet. Bokföringsprogram, bankkonton och pennor. Bör finnas som en budgetpost och därmed också kostnadsställe.\\
	\multicolumn{3}{l}{10--19 Verksamheten}\\
	10				&	Lokalen\\
	11				&	Värkstaden\\
	12				&	Labbet\\
	13				&	Amatörradio\\
	19				&	Övrig verksamhet					& Övrigt direkt relaterat till verksamheten.\\
	\multicolumn{3}{l}{20--29 Auktionen}\\
	20				&	Auktion								& Allt som rör auktionen. Vill man budgetera på mer detaljnivå kan man lägga till konton för till exempel sorteringar, arbete innan och liknande.\\
	\multicolumn{3}{l}{30--39 Inkomster}\\
	30				&	Kylen								& Den löpande försäljningen av fika och kylvaror.\\
	31				&	Öppen försäljning lokalen			& ”Öppen försäljning”, alltså försäljning där medlemmarna själva tar varan och betalar. Intressant med ett eget kostnadsställe för att räkna på svinn.\\
	32				&	Övrig försäljning					& All annan försäljning som \acr{eta} gör.\\
	\multicolumn{3}{l}{40--49 Övriga kostnader}\\
	40				&	Övriga kostnader					& För budgetposten ”övriga kostnader”.
	\multicolumn{3}{l}{50--79 \emph{Ej använda}}\\
	\multicolumn{3}{l}{80--99 Rörande projekt}\\
	80				&	ETAprojekt							& Projekt som inte kan debiteras ett verksamhetsområde.\\
	81--99			&	\emph{Reserverat för specifika projekt} & Reserverat för specifika projekt.\\
\end{longtable}