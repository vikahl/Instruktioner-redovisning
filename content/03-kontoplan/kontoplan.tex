Kontoplanen som redovisas här är ett förslag på kontoplan baserad på primärt \acr{bas}~2014 kontoplan 1 med stöd av kontoplan 2 där det behövts. \footnote{Kontoplanerna finns att tillgå gratis på \acr{bas}-gruppens hemsida, \url{www.bas.se}}

Alla konton i \acr{bas}-kontoplanen är inte användbara för \acr{eta} och det är generellt sällan man använder kontoplanen rent av. 
Därför har vissa konton lagts till och många har inte tagits med. En del konton har bytt namn, och i en del fall så står båda namnen med i listan nedan \todo{Varför?}. Med största sannolikhet kommer det komma situationer där ytterligare konton behövs, men i så fall är det bara att skapa dessa på rätt plats i listan. Ta i sådant fall hjälp av \acr{bas} kontoplanen om osäkerhet finns över var de passar in.


\section{Tillgångskonton}
\todo{Eftersom man inte kan ha marginalanteckningar i tabeller står 'todonotes' nedan i kursiv stil.}
\begin{minipage}{\fullwidthlength}
	\begin{longtable}[l]{l p{0.4\linewidth} p{0.5\linewidth}}
		Konto	&	Namn								& Kommentar \\ \toprule \endhead
		1510	&	Fordringar medlemmar \newline \emph{Kundfordringar} & Pengar som medlemmar är skyldiga \acr{eta}\\
		1515	&	Osäkra fordringar medlemmar \newline \emph{Osäkra kundfordringar}	& Pengar som medlemmar är skyldiga men där det är osäkert om man får igen. \newline Se ”Farnell” för exempel.\\
		1600	&	Övriga fordringar					& Andra personer än medlemmar som är skyldiga föreningen pengar.\\
		1910	&	Kassa								& För kassorna som förvaras i lokalen\\
		1911	&	Lokalkassa							& De olika kassakontona som ligger framme i lokalen.\\
		1930	&	Checkkonto \newline \emph{Företagskonto/checkkonto/affärskonto}	& Bankkonto\\
	\end{longtable}
\end{minipage}

\section{Skuldkonton}
\begin{minipage}{\fullwidthlength}
	\begin{longtable}[l]{l p{0.4\linewidth} p{0.5\linewidth}}
		Konto	&	Namn								& Kommentar \\ \toprule \endhead
		2440	&	Leverantörsskulder					& Skulder föreningen har till medlemmar eller leverantörer. \\
		2999	&	OBS-konto							& I den löpande bokföringen händer det ibland att det saknas underlag för en transaktion eller att det inte går att avgöra hur den ska bokföras. För att inte bokföringsarbetet ska avstanna använder men ett observationskonto som man sedan bokför från när underlaget är klart. Det här kontot ska nollställas vid årsbokslut. \emph{Se 'OBS-konto'}.
	\end{longtable}
\end{minipage}


\section{Intäktskonton}
\begin{minipage}{\fullwidthlength}
	\begin{longtable}[l]{l p{0.4\linewidth} p{0.5\linewidth}}
		Konto	&	Namn								& Kommentar \\ \toprule \endhead
		3100	&	Medlemsavgifter \newline \emph{Momsfria intäkter} \\
		3610	&	Försäljning av varor				& Försäljning av material och materiel i lokalen. Inkluderar kyl, laminat och liknande varor som köps in och sedan säljs efter hand.\\
		3611	&	Förmedling av varor					& Förmedling av varor, Farnell-beställningar, Atmel och dylikt\\
		3612	&	Försäljning auktion \emph{Behöver detta ett eget konto? Egentligen inte, men det kanske underlättar? Försäljningen här har ju väldigt lite att göra med resterande försäljning och för att särskilja auktionskafé och utrop kanske detta behövs?} & Auktionen\\
		3740	&	Öres-- och kronutjämning			& Specialkonto som används för öresutjämning\\
		3900	&	Övriga intäkter						& För intäkter som inte täcks av något annat konto
	\end{longtable}
\end{minipage}

\section{Kostnadskonton}
\begin{minipage}{\fullwidthlength}
	\begin{longtable}[l]{l p{0.4\linewidth} p{0.5\linewidth}}
		Konto	&	Namn								& Kommentar \\ \toprule \endhead
		4000	&	Inköp av varor						& Inköp av varor som säljs i lokalen, t.ex. kyl, laminat och dylikt.\\
		5000	&	Lokalkostnader						& Utgifter tillhörande lokalerna, fasta installationer.\\
		5400	&	Förbrukningsmateriel				& Förbrukningsmateriel, lödtenn, pennor och papper.\\
		5410	&	Inventarier \newline \emph{Förbrukningsinventarier} & Inköp av inventarier som inte ska skrivas av\emph{Skriv varför}\\
		5700	&	Frakter och transporter				& Alla kostnader för transporter, t.ex. auktionshämtning. Slår ihop kontogrupperna 56 och 57 då behov inte anses finnas för att särredovisa egen transport (t.ex. drivmedel) och externt inköpt transport (frakt) i två olika grupper.\\
		5900	&	Reklam och PR						& All form av medlemsrekryterande verksamhet.\\
		6000	&	Övriga försäljningskostnader		& Kostnader som uppstår i samband med försäljning av varor och förmedling av sådana.\\
		6210	&	Telekommunikation					& Telefonen som står i lokalen.\\
		6250	&	Postbefordran						& Post och porto\\
		6310	&	Försäkringar						& \acr{Eta}s försäkringar.\\
		6570	&	Bankkostnader						& Bankkonto kostar pengar.\\
		6900	&	Övriga kostnader					& Det kommer alltid finnas kostnader som inte täcks av de andra kontona.\\
		6970	&	Tidningar, tidskrifter och facklitteratur & Biblioteket, Kalle Anka och liknande.\\
		7600	&	Förmåner medlemmar					& Allt som ges till medlemmarna, kaffe, tisdagsfika och medlemsgrillningar.\\[1.5\baselineskip]
		\multicolumn{3}{l}{Finansiella och andra inkomster/intäkter och utgifter/kostnader}\\ \midrule
		8310	&	Ränteintäkter						& Räntan som fås på t.ex. bankkontot.\\
		8999	&	Årets resultat\emph{Ordna med motkonto. Förklara.}\\
	\end{longtable}
\end{minipage}