
				\section{Auktionshämtning}
				\emph{se transporter}
			
				\section{Auktionssortering, mat}
				\emph{se medlemsförmåner}
			
				\section{Förbrukningsinventarier}
				\emph{se inventarier}
			
				\section{Förbrukningsmateriel}
				
				I förbrukningsmateriel ryms all inköpt materiel som inte har ett bestående värde och som förr eller senare förbrukas.
		
		Här ingår alltså till exempel pennor och papper, men även lödtenn och komponenter. När det gäller verktyg och gerät får en bedömning från fall till fall göras. Man kan anse att borr som köps in är förbrukningsmateriel eftersom det inte har något bestående värde och slits ut relativt snabbt. När det gäller borrmaskiner däremot, kan man anse att de är tänkta att användas under en längre tid och därmed bör räknas som inventarier. Om osäkerhet uppstår är det viktigare att kostnaden hamnar på rätt kostnadsställe snarare än om det är förbrukningsmateriel eller en inventarie.
			
				\begin{redovisning}
					Redovisning : Förbrukningsmateriel bokförs på konto \kref{5400} och tillhörande kostnadsställe.
				\end{redovisning}
			
				\begin{bokslut}
					Bokslut : 
				\end{bokslut}
			
				\subsection{Kommentar}
				
				\section{Gerät}
				\emph{se inventarier}
			
				\section{Instrument}
				\emph{se inventarier}
			
				\section{Inventarier}
				
				lalala
			
				\begin{redovisning}
					Redovisning : 
				\end{redovisning}
			
				\begin{bokslut}
					Bokslut : 
				\end{bokslut}
			
				\subsection{Kommentar}
				
				\section{Mat}
				\emph{se medlemsförmåner}
			
				\section{Medlemsförmåner}
				
				Medlemmar i \acr{eta} har flera förmåner efter att man har gått med i föreningen. Man har tillgång till komponentförrådet, lokalen och har också tillgång till kostnadsfritt kaffe och tisdagsfika.
		
		Utgifter för komponentbaren och lokalen bokförs på sina respektive konton, under medlemsförmåner räknas de förmåner som inte strikt har med verksamheten. Här innefattas bland annat \emph{tisdagsfika, fritt kaffe, mat i samband med sorteringar} och \emph{mutgodis}. Exakt vad gränsen går mellan vanlig verksamhet och förmåner är inte entydigt definierat utan kassören får göra en bedömning i gränsfallen.
		
			
				\begin{redovisning}
					Redovisning : Medlemsförmåner bokförs på konto \kref{7600} med tillhörande kostnadsställe.
				\end{redovisning}
			
				\begin{bokslut}
					Bokslut : 
				\end{bokslut}
			
				\subsection{Kommentar}
				Man skulle kunna kalla den här för \emph{föreningskvällar} som motsvarande kostnader har bokförts på tidigare. Det gör det dock inte tydligt att även kaffe, mutgodis och liknande bör hamna på det här kontot. Det är trots allt intressant att följa upp hur mycket pengar som läggs på förmåner till medlemmarna (till exempel kaffe).
				\section{Mutgodis}
				\emph{se medlemsförmåner}
			
				\section{Tisdagsfika}
				\emph{se medlemsförmåner}
			
				\section{Transporter}
				
				Speciellt i samband med auktionen förekommer det många transporter och därmed också utgifter för dessa. Transporter av varor förekommer även under resten av året, då det till exempel handlas till kylen.
		
		Dessa transporter varierar mellan att vara extern inhyrda fordon, fordon hyrda av kåren och medlemmars egna fordon som lånas ut. I kontoplanen finns ett konto för alla typer av transporter, oavsett om det är kostnader för egna transporter eller inhyrt fraktbolag.
		
		I transportkostnaden inräknas alla utgifter som direkt rör transporten, till exempel drivmedel, parkeringskostnader eller hyra av fordon.
		
		Notera att post inte räknas som transport utan istället \kref{6250}.
		
			
				\begin{redovisning}
					Redovisning : Transporter bokförs på kostnadskonto \kref{5700} med ett kostnadsställe relevant för aktiviteten. Är det en transport som inte rör någon specifik kategori kan kostnadsställe utelämnas, men är det till exempel inköp av kylvaror ska dessa noteras på rätt kostnadsställe.
		
		Om transporten rör flera kostnadsställen kan antingen kostnaden delas upp eller bokföras på ett av dessa, beroende på hur stor kostnaden är.
				\end{redovisning}
			
				\begin{bokslut}
					Bokslut : 
				\end{bokslut}
			
				\subsection{Kommentar}
				