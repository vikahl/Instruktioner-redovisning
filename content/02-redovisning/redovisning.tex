
				\section{Auktionshämtning}
				\emph{se transporter}
			
				\section{Auktionssortering, mat}
				\emph{se medlemsförmåner}
			
				\section{Bestick}
				\emph{se förbrukningsmateriel}
			
				\section{Diskmedel}
				\emph{se förbrukningsmateriel}
			
				\section{Farnell}
				\emph{se förmedling av varor}
			
				\section{Förbrukningsinventarier}
				\emph{se inventarier}
			
				\section{Förbrukningsmateriel}
				
				I förbrukningsmateriel ryms all inköpt materiel som inte har ett bestående värde och som förr eller senare förbrukas.
		
		Här ingår alltså till exempel pennor och papper, men även lödtenn och komponenter. När det gäller verktyg och gerät får en bedömning från fall till fall göras. Man kan anse att borr som köps in är förbrukningsmateriel eftersom det inte har något bestående värde och slits ut relativt snabbt. När det gäller borrmaskiner däremot, kan man anse att de är tänkta att användas under en längre tid och därmed bör räknas som inventarier. Om osäkerhet uppstår är det viktigare att kostnaden hamnar på rätt kostnadsställe snarare än om det är förbrukningsmateriel eller en inventarie.
		
		Förbrukningsmateriel som har en mer permanent karaktär och är tydligt kopplat till lokalen kan istället bokföras på kontot för lokalkostnader.
		
					\begin{redovisning}
						Förbrukningsmateriel bokförs på konto \kref{5400} och tillhörande kostnadsställe.
					\end{redovisning}
				
				\section{Förmedling av varor}
				
				
				\section{Gerät}
				\emph{se inventarier}
			
				\section{Instrument}
				\emph{se inventarier}
			
				\section{Inventarier}
				
				
					\todo{Får ETA verkligen skriva av allt direkt? Hitta en tillförlitlig källa på det!}
				När man köper in inventarier redovisas de normalt som anläggningstillgångar och skrivs av under sin livslängd. För förbrukningsinventarier, inventarier av mindre värde\footnote{Anskaffningsvärde exklusive moms som understiger ett halvt prisbasbelopp, 22\,000~kr (2014)} och korttidsinventarier, kan utgiften istället dras av direkt, så kallat direktavdrag. Inventarier som har ett naturligt samband ska bedömas gemensamt, samma sak gäller för inventarier som är en del av en större inventarieinvestering. Man kan alltså inte 'dela upp' inventarierna för att få ner värdet.
		
					\begin{redovisning}
						
		För inventarier som understiger ovan nämnda värde bokförs kostnaden direkt på kontot för inventarier. För inventarier som har en tydlig koppling till lokalen, till exempel lysrör eller elcentral till förrådet, kan istället bokföringen ske på kontot för lokalutgifter.
		
		Glöm inte att ange rätt kostnadsställe för inventarierna.
		
					\end{redovisning}
				
				\section{Mat}
				\emph{se medlemsförmåner}
			
				\section{Medlemsavgifter}
				
				Medlemsavgifter utgör huvudinkomsten för \acr{eta} och bokförs på därtill förenligt intäktskonto.
		
		Det förekommer ibland att medlemmar betalar för flera år samtidigt och man kan då diskutera om man ska periodisera intäkten över de åren eller om hela intäkten ska bokföras direkt. Detta förekommer dock ganska sällan och påverkar inte resultatet nämnvärt. Det finns dessutom risk att informationen inte förs vidare till nästa års styrelse så eventuella vinningar är mindre än de eventuella extra komplikationerna.
		
		Eftersom medlemsavgifterna inte följer verksamhetsår kan man på samma sätt diskutera huruvida man ska periodisera alla avgifter över årsbrytet, men då antalet medlemmar är ganska konstant medför det här, precis som ovan nämnda punkt, mycket extraarbete för lite resultat.
		
					\begin{redovisning}
						Medlemsavgifter krediteras kontot \kref{3100}. Kostnadsställe behöver inte anges eftersom det bara finns en typ av inkomst.
					\end{redovisning}
				
				\section{Medlemsförmåner}
				
				
					\todo{Man skulle kunna kalla den här för \emph{föreningskvällar} som motsvarande kostnader har bokförts på tidigare. Det gör det dock inte tydligt att även kaffe, mutgodis och liknande bör hamna på det här kontot. Det är trots allt intressant att följa upp hur mycket pengar som läggs på förmåner till medlemmarna (till exempel kaffe).}
				Medlemmar i \acr{eta} har flera förmåner efter att man har gått med i föreningen. Man har tillgång till komponentförrådet, lokalen och har också tillgång till kostnadsfritt kaffe och tisdagsfika.
		
		Utgifter för komponentbaren och lokalen bokförs på sina respektive konton, under medlemsförmåner räknas de förmåner som inte strikt har med verksamheten. Här innefattas bland annat \emph{tisdagsfika, fritt kaffe, mat i samband med sorteringar} och \emph{mutgodis}. Exakt vad gränsen går mellan vanlig verksamhet och förmåner är inte entydigt definierat utan kassören får göra en bedömning i gränsfallen.
		
					\begin{redovisning}
						Medlemsförmåner bokförs på konto \kref{7600} med tillhörande kostnadsställe.
					\end{redovisning}
				
				\section{Mutgodis}
				\emph{se medlemsförmåner}
			
				\section{OBS-konto}
				
				I arbetet med bokföringen händer det ibland att det saknas underlag för en transaktion eller att det inte går att avgöra hur den ska bokföras. Gängse bokföringsregler \todo{källa behövs} säger att transaktioner ska bokföras i den ordning de har uppstått. För att inte bokföringsabetet ska avstanna helt när vid tidigare nämnda situationer kan ett \acr{obs}-konto, observationskonto, användas för att bokföra kostnaden på. Man debiterar då kostnaden på \acr{obs}-kontot och när underlaget sedan har kommit fram krediterar man \acr{obs}-kontot och debiterar rätt konto.
		
		Vid årsbokslutet ska saldot på \acr{obs}-kontot vara noll.
		
					\begin{redovisning}
						Okända kostnader debiteras kontot \kref{2999} och när sedan kostnaden reds ut krediteras \kref{2999} och rätt konto debiteras.
					\end{redovisning}
				
					\begin{bokslut}
						Vid verksamhetsårets slut måste saldot på \acr{obs}-kontot vara noll. Om det finns kostnader man inte kan reda ut får en diskussion med revisorerna och styret tas för att komma fram hur det ska bokföras. Saldot \kref{6900} kan användas för kostnader som inte passar in på något annat konto.
					\end{bokslut}
				
				\section{Tallrikar}
				\emph{se förbrukningsmateriel}
			
				\section{Tisdagsfika}
				\emph{se medlemsförmåner}
			
				\section{Transporter}
				
				Speciellt i samband med auktionen förekommer det många transporter och därmed också utgifter för dessa. Transporter av varor förekommer även under resten av året, då det till exempel handlas till kylen.
		
		Dessa transporter varierar mellan att vara extern inhyrda fordon, fordon hyrda av kåren och medlemmars egna fordon som lånas ut. I kontoplanen finns ett konto för alla typer av transporter, oavsett om det är kostnader för egna transporter eller inhyrt fraktbolag.
		
		I transportkostnaden inräknas alla utgifter som direkt rör transporten, till exempel drivmedel, parkeringskostnader eller hyra av fordon.
		
		Notera att post inte räknas som transport utan istället \kref{6250}.
		
					\begin{redovisning}
						Transporter bokförs på kostnadskonto \kref{5700} med ett kostnadsställe relevant för aktiviteten. Är det en transport som inte rör någon specifik kategori kan kostnadsställe utelämnas, men är det till exempel inköp av kylvaror ska dessa noteras på rätt kostnadsställe.
		
		Om transporten rör flera kostnadsställen kan antingen kostnaden delas upp eller bokföras på ett av dessa, beroende på hur stor kostnaden är.
					\end{redovisning}
				