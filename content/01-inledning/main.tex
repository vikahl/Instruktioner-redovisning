\chapter{En introduktion till bokföring}
Bokföring, att föra sina böcker, handlar om att föra en lista över alla affärshändelser som har hänt. Oavsett om det handlar om att man har köpt några pennor, tagit lån, investerat i maskiner eller betalat ut löner så ska detta föras upp i bolagets böcker.

Bokföringen finns till för att man ska kunna ha kontroll över sin ekonomi och kunna följa upp hur utvecklingen går. För en förening som \acr{eta} ligger också den löpande bokföringen till grund för de ekonomiska rapporterna som ska lämnas till föreningsstämman och kårens revisorer. Eftersom \acr{eta} är en förening med liten omsättning finns det inga lagkrav på att föra bok,\footnote{Föreningar som uppfyller något av nedanstående är skyldiga att föra bok.
	\begin{itemize}\itemsep1pt
		\item Tillgångarnas marknadsvärde överstiger 1,5 miljoner
		\item Föreningen bedriver närings\-verksamhet
		\item Föreningen är moderföretag i en koncern
	\end{itemize}
}
men utgifter ska redovisas och det rekommenderas att följa gängse bokföringsrutiner när redovisningen upprättas.

Eftersom det inte finns något lagstadgat bokföringskrav kan man undgå från en del av reglerna för att bokföringen ska underlättas. Man kan slå ihop transaktionerna och till exempel bara redovisa saldot i kassorna månadsvis. Huvudtanken bör dock vara att följa god redovisningssed i den mån det går.
% Fotnoter från http://wiki.sverok.se/wiki/Bokf%C3%B6ring,_introduktion

\section{Dubbel bokföring}
Den enklaste formen av bokföring är enkel bokföring, där varje transaktion skrivs upp på ett ställe. Köper man till exempel lödtenn för 100~kr skriver man upp lödtenn i kategorin \emph{Förbrukningsmateriel labb}. En personlig kassabok fungerar oftast på det här sättet.
Vid dubbel bokföring skrivs varje transaktion upp två gånger, man beskriver både var pengarna kommer ifrån och var de går. Med samma inköp på lödtenn för 100~kr hade man i dubbel bokföring noterat att pengarna kom från bankkontot och gick till förbrukningsmateriel.

Dubbel bokföring gör det mycket enklare att upptäcka fel och att göra avstämningar mellan olika konton. I stort sett samtliga affärssystem som finns för bokföring använder dubbel bokföring och i den resterande delen av den här texten kommer bara dubbel bokföring att användas.

\section{Verifikationer}
För varje affärstransaktion ska det finnas ett underlag som beskriver händelsen. Oftast är detta ett kvitto vid inköp, ett kontoutdrag eller en faktura. Den här handlingen kallas verifikation och för varje transaktion ska det finnas ett verifikat i bokföringen i den ordningen som transaktionerna ägt rum.

Finns det inga externa underlag som kvitton, till exempel vid bokföringsmässiga transaktioner kan man skriva en bokföringsorder.\todo{Här ska det in text}

\section{Konton}
När man pratar om konton i bokföring menar man inte alltid bankkonton. Ett konto i bokföringen är en redovisningspost, en kategori, snarare än ett konto på banken. Visserligen har föreningens bankkonto även ett konto i bokföringen men även till exempel förbrukningsinventarier, medlemsavgifter och inventarier har konton i bokföringen.

Man delar upp kontona i fyra olika kategorier, olika kontoslag:
\begin{itemize}\itemsep2pt
	\item tillgångar
	\item skulder
	\item intäkter
	\item kostnader
\end{itemize}

\clearpage

\subsection{Tillgångskonton}
\marginnote[]{
Konto: 1\,000--1\,999\\
\small
\begin{tabular}{ll}
	\textbf{debet}	& \emph{pengar in}\\
					& tillgångsökning\\
	\textbf{kredit} & \emph{pengar ut}\\
					& tillgångsminskning 
\end{tabular}
}
Tillgångskontona används för att bokföra sådant som föreningen äger. Detta innefattar dels likvida medel som pengar på bankkontot, kylkassan och liknande. I den mån föreningen äger inventarier eller maskiner som har ett bokfört värde så bokförs även det på bokföringskonto.

Är medlemmar eller andra personer skyldiga pengar har föreningen en fordran och detta skrivs upp som en tillgång. Om föreningen istället är skyldig en person pengar är detta en skuld, inte en negativ fordran, och ska bokföras på skuldkonton.

\subsection{Skuldkonton}
\marginnote[]{
Konto: 2\,000--2\,999\\
\small
\begin{tabular}{ll}
	\textbf{debet}	& \emph{skuld minskar}\\
	\textbf{kredit}	& \emph{skuld ökar}
\end{tabular}
}
Skuldkontona används för de skulder föreningen har. Detta kan dels innehålla skulder till banker och leverantörer så väl som skulder till medlemmar. I företag går ägarna in med pengar när man startar företaget och företaget har då en skuld till ägarna. \acr{eta} har tidigare inte haft några skulder.

\subsection{Intäktskonton}
\marginnote[]{
Konto: 3\,000--3\,999 (8\,000)\\
\small
\begin{tabular}{ll}
	\textbf{debet}	& \emph{felkorrigering}	\\
					& inkomstminskning\\
	\textbf{kredit}	& \emph{får pengar}\\
					& inkomstökning
\end{tabular}
}
Medlemsavgifter, försäljning av mat och auktionsgods är exempel på intäkter till föreningen som bokförs på 3\,000-konton. Dessa konton används bara när föreningen har en faktiskt intäkt, om pengar flyttas från kassan till bankkontot är det ingen intäkt utan bara en flytt mellan två tillgångskonton.

Man pratar sällan om negativa intäkter och debet-delen används nästan enbart för felkorrigeringar i det här kontoslaget.

\subsection{Kostnadskonton}
\marginnote[]{
Konto: 4\,000--8\,999\\
\small
\begin{tabular}{ll}
	\textbf{debet}	& \emph{varor köps}\\
					& utgiftsökning\\
	\textbf{kredit} & \emph{felkorrigering}\\
					& utgiftsminskning
\end{tabular}
}
Kostnader är allt ifrån inköp av gem och pennor så väl som gas till svetsen eller datorutrustning. Alla inköp som inte har ett bestående värde och är att betrakta som en investering är kostnader för föreningen. \emph{Bestående värde} är inte helt entydigt definierat, men om varan kan säljas vidare kan man anta att den har ett bestående värde. Värt att notera är att normalt bokförs till exempel datorutrustning som förbrukningsmateriel och har inte något bestående värde på samma sätt som dyrare maskiner.

\acr{eta} har tidigare inte gjort några avskrivningar på maskiner och har inte heller krav på sig att göra så. \todo{tror jag i alla fall} Därför kan kostnadskonton används för nästan alla inköp.

\subsection{Kontoplan}
En kontoplan är en lista med konton som har dels ett nummer och ett namn. Det finns flera standardiserade kontoplaner, till exempel \acr{bas} eller \acr{eu}. 
Även om man kan hitta på en helt egen kontoplan så är det vanligaste att man följer en standardiserad plan och sedan lägger till eller tar bort konton som man behöver. \acr{eta} har till exempel inga anställda och äger inte heller fastigheter, så konton som rör det behövs inte.

Kontoplanen som är föreslagen i den här skriften är baserad på \acr{bas}-kontoplanen och sedan anpassad för \acr{eta}s behov. Detta gör det lätt att sätta sig och anpassa kontoplanen om behov uppstår.

\section{Kostnadsställe}
Ett kostnadsställe är en avgränsad enhet eller verksamhet inom organisationen och på \acr{eta} kan det till exempel vara Värkstaden, labbet eller auktionen. Det finns inget krav på att använda kostnadsställen och det går utmärkt att bokföra utan dessa. De kan dock underlätta att gruppera utgifter i organisatoriska grupper också och inte bara efter kontotyp, speciellt när man ska göra budgetuppföljning. 

När man använder kostnadsställen anger man dels kostnadskonto och dels kostnadsställe.

Kostnadsställen kan ibland ha ett annat namn i redovisningsprogrammet, i Visma administration heter det till exempel resultatenhet.

\section{Kontant-- eller fakturametoden}
I större företag används oftast fakturametoden för att bokföra fakturor. Metoden innebär att fakturor bokförs två gånger, dels när fakturan tas emot och dels när den betalas. När fakturan tas emot bokförs den som en skuld eller fordran och när fakturan faktiskt är betald bokför man bort skulden mot bankkontot.

Mindre företag och föreningar \footnote{Nettoomsättning under 3 miljoner kronor och utan krav på att upprätta årsbokslut} kan istället använda kontantmetoden. Då bokförs fakturorna först då de betalas och man noterar alltså inte att föreningen har en skuld eller fordran. Detta underlättar bokföringen, men gör att man inte har samma insyn i nuläget. För \acr{eta} är inte värdet på fakturorna så stort att behov av större insyn finns. 


\section{Bokföringsexempel}
\todo{Uppdatera det här avsnittet med rätt kontonummer och kostnadsställen}
Nedan följer några exempel på verifikat och bokförda transaktioner.

\subsection{Inköp av förbrukningsmateriel}
I det här exemplet har det köps in lödtenn och annan förbrukningsmateriel för att använda i labbet.
Betalningen har skett med kontokort och totala summan av de inköpta varorna är 285~kr. Då sakerna ska användas i labbet anger vi också labbets kostnadsställe.

I och med den här betalningen har saldot på bankkontot minskats, och bankkontot ska därför krediteras. För att summan ska bli lika stora i båda kolumnerna måste kostnadskontot således debeteras.

\begin{longtable}{llrr}
	\caption{Exempel: Bokföra materielinköp}\\
	Konto	& Kontonamn					& Debet		& Kredit\\ \toprule
	1930	& Företagskonto				& 			& 285~kr\\
	5410.10	& Förbrukningsinventarier	& 285~kr	& \\ \bottomrule
			& Summa						& 285~kr	& 285~kr
\end{longtable}

\subsection{Inbetalning av medlemsavgift}
Nu har \acr{eta} istället fått pengar i form av en medlem som har betalat in sin medlemsavgift. Då \acr{eta} bara har en typ av medlemsavgift finns det ingen anledning att särskilja på dessa med till exempel kostnadsställen.

I det här exemplet sätts pengar in på bankkontot, och det ska alltså debeteras. För att summan av debet och kredit ska bli lika stora måste således kontot för medlemsavgifter krediteras.

\begin{longtable}{llrr}
	\caption{Exempel: Bokföra medlemsavgift}\\
	Konto	& Kontonamn					& Debet		& Kredit\\ \toprule
	1930	& Företagskonto				& 100~kr	& \\
	3210	& Medlemsavgifter			& 			& 100~kr\\ \bottomrule
			& Summa						& 100~kr	& 100~kr
\end{longtable}

\subsection{Inköp av hylla till förrådet}
% I det här fallet behövs en ny hylla till förrådet. När man köper in inventarier redovisas de normalt som tillgångar och skrivs av under sin livslängd. För inventarier av mindre värde\footnote{Anskaffningsvärde som understiger ett halvt prisbasbelopp, 22\,000~kr (2014)} kan utgiften istället kostnadsföras som förbrukningsinventarier. \acr{Eta} får dock alltid bokföra inventarieinköp som direkta kostnader, men det kan vara bra att känna till de allmänna principerna också.

Hyllan i exemplet kostade 5\,200~kr och har betalats med kontokort. Saldot på bankkontot minskade, alltså ska det kontot debeteras.

\begin{longtable}{llrr}
	\caption{Exempel: Bokföra inventarieinköp}\\
	Konto	& Kontonamn					& Debet		& Kredit\\ \toprule
	1930	& Företagskonto				& 			& 5\,200~kr\\
	5411.20	& Förbrukningsinventarier	& 5\,200~kr	& \\ \bottomrule
			& Summa						& 5\,200~kr	& 5\,200~kr
\end{longtable}

\subsection{Inköp till kylen}
% När varor till kylen köps in får \acr{eta} ett varulager som har ett värde. Man kan tänka sig att man bokför varuinköpet som en kostnad och sedan varuförsäljningen som en inkomst, men det är då svårt att se om man

När det gäller inköp av varor kan de bokföras på olika sätt. I ett företag hade man bokfört inköpet som att värdet på varulagret ökade, men för \acr{eta} tror jag det blir för svårt att hålla i en rutin med varulager när det gäller kylarna. De fungerar ju på sådant sätt att medlemmarna själva betalar och det är svårt att veta exakt vad som säljs (även om man kan räkna på lagervärde). Ibland ges delar av lagret bort (mutgodis) och då måste det bokföras, vilket antagligen blir onödigt komplicerat.

Därför bokförs inköpen till kylen som en kostnad, och när försäljningen sedan sker uppstår en inkomst.
\begin{longtable}{llrr}
	\caption{Exempel: Bokföra inventarieinköp}\\
	Konto	& Kontonamn					& Debet		& Kredit\\ \toprule
	1930	& Företagskonto				& 			& 7\,869~kr\\
	4011	& Inköp av varor			& 7\,869~kr	& \\ \bottomrule
			& Summa						& 7\,869~kr	& 7\,869~kr
\end{longtable}


\subsection{Försäljning via kylen}



\subsection{Insättning av kylkassa}